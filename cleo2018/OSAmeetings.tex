%%%%%%%%%%%%%%%%%%%%%%%%%%%%%%%%%%%%%%%%%%%%%%%%%%%%%%%
%                   File: OSAmeetings.tex             %
%                  Date: March 21, 2007  MSD          %
%                                                     %
%     For preparing LaTeX manuscripts for submission  %
%       submission to OSA meetings and conferences    %
%                                                     %
%       (c) 2007 Optical Society of America           %
%%%%%%%%%%%%%%%%%%%%%%%%%%%%%%%%%%%%%%%%%%%%%%%%%%%%%%%

\documentclass[letterpaper,10pt]{article}
\usepackage{osameet2}

%% standard packages and arguments should be modified as needed

\usepackage{amsmath,amssymb}

%\usepackage[pdftex,colorlinks=true,bookmarks=false,citecolor=blue,urlcolor=blue]{hyperref} %pdflatex
\usepackage[dvips,colorlinks=true,bookmarks=false,citecolor=blue,urlcolor=blue]{hyperref} %latex w/dvips

\begin{document}

\title{Preparing a Manuscript for Submission to an Optical Society of America Meeting or Conference}

\author{Author name(s)}
\address{Author affiliation and full address}
\email{e-mail address}

\begin{abstract}
\LaTeX{} users preparing manuscripts for OSA meetings or conferences
should use the new \texttt{osameet2.sty} style file and should observe the
guidelines described here to adhere to
OSA requirements. Users of Bib\TeX{} may use the \texttt{osajnl.bst}
style file, which is included in this distribution. Comments and
questions should be directed to the OSA Technical Papers Department (tel: +1 202.416.6191,  e-mail: cstech@osa.org).
\end{abstract}

\ocis{000.0000, 999.9999.}


\section{Main Text}

\subsection{Typographical Style}
Margins and type size will be set by the OSA \LaTeX{}
commands for title, author names and addresses, abstract,
references, captions, and so on. The \texttt{osameet2.sty} package
references \texttt{mathptmx.sty} for Times text and math fonts.
Authors who require Computer Modern font may modify the style file
or, preferably, invoke the package \texttt{ae.sty} or similar for
optimum output with Computer Modern.

\subsection{Author Names and Affiliations}
Author names should be given in full with first initials spelled out to assist with indexing.
Affiliations should follow the format division, organization, and address---and complete postal information should be given.
Abbreviations should not be used. United States addresses should end
with ``, USA.''

\subsection{Abstract} The abstract
should be limited to no more than words. It should be an
explicit summary of the paper that states the problem, the methods
used, and the major results and conclusions. If another publication author is referenced in the abstract, abbreviated information
(e.g., journal, volume number, first page, year) must be
given in the abstract itself, without a reference number. (The item referenced in the abstract should be the first
cited reference  in the body.)

\subsection{OCIS Subject Classification} Two Optics Classification and
Indexing Scheme (OCIS) subject classifications should be placed at
the end of the abstract with the \verb+\ocis{}+ command. OCIS
codes can be found at
\href{http://www.osapublishing.org/submit/ocis/}{http://www.osapublishing.org/submit/ocis/}.

\subsection{Notation}
\subsubsection{General Notation}
Notation must be
legible, clear, compact, and consistent with standard usage. In
general, acronyms should be defined at first use.

\clearpage

\subsubsection{Math Notation}
Equations should use standard \LaTeX{} or AMS\LaTeX{} commands (sample from Krishnan \textit{et al.} \cite{krishnan00}).

\begin{eqnarray}
\bar\varepsilon &=& \frac{\int_0�\infty\varepsilon
\exp(-\beta\varepsilon)\,{\rm d}\varepsilon}{\int_0�\infty
\exp(-\beta\varepsilon)\,{\rm d}\varepsilon}\nonumber\\
&=& -\frac{{\rm d}}{{\rm d}\beta}\log\Biggl[\int_0�\infty\exp
(-\beta\varepsilon)\,{\rm d}\varepsilon\Biggr]=\frac1\beta=kT.
\end{eqnarray}

\section{Tables and Figures}
Figures and illustrations should be incorporated directly into the
manuscript, and the size of a figure should be commensurate with the amount
and value of the information conveyed by the figure.

\begin{figure}[htbp]
  \centering
  \includegraphics[width=8.3cm]{OT10000F1}
\caption{Sample figure with preferred style for labeling parts.}
\end{figure}


\begin{table}[htb]
 \centering \caption{Sample Table}
\begin{tabular}{ccc}
    \hline
    One & Two & Three \\
    \hline
    Eins & Zwei & Drei \\
    Un & Deux & Trois \\
    Jeden & Dv\v{e} & T\v{r}i \\
    \hline
   \end{tabular}
    \end{table}

 No more than
three figures should generally be included in the paper.
Place figures as close as possible to where they
are mentioned in the text. No part of a figure should extend beyond text width, and text should not wrap around figures.




\section{References}
References should be cited with the \verb+\cite{}+ command.
Bracketed citation style, as opposed to superscript, is preferred
\cite{krishnan00,vantrigt97,masters93,shoop97,kalman76,craig96,steup96}.
The \texttt{osameet2.sty} style file references \texttt{cite.sty}. Comprehensive journal abbreviations are available on the CrossRef web site:
\href{http://www.crossref.org/titleList/}{http://www.crossref.org/titleList/}.

\begin{thebibliography}{99}

\bibitem{krishnan00} E. Krishnan, A. M. Shan, T. Rishi, L. A. Ajith, C. V.
Radhakrishnan, \textit{On-line Tutorial on \LaTeX{}},
``Mathematics'' (Indian \TeX{} Users Group, 2000), \\
\url{http://www.tug.org/tutorials/tugindia/chap11-scr.pdf}.

\bibitem{vantrigt97} C. van Trigt, ``Visual system-response functions and estimating reflectance,''
J. Opt. Soc. Am. A \textbf{14}, 741--755 (1997).

\bibitem{masters93} T. Masters, \emph{Practical Neural Network Recipes in C++} (Academic, 1993).

\bibitem{shoop97} B. L. Shoop, A. H. Sayles, and D. M. Litynski, ``New devices for optoelectronics: smart pixels,''
in \emph{Handbook of Fiber Optic Data Communications},
C. DeCusatis, D. Clement, E. Maass, and R. Lasky, eds. (Academic, 1997), pp. 705--758.

\bibitem{kalman76} R. E. Kalman,``Algebraic aspects of the generalized inverse of a rectangular matrix,'' in
\emph{Proceedings of Advanced Seminar on Generalized Inverse and Applications}, M. Z. Nashed, ed. (Academic, 1976), pp. 111--124.

\bibitem{craig96} R. Craig and B. Gignac, ``High-power 980-nm pump lasers,''
in \emph{Optical Fiber Communication Conference}, Vol. 2 of 1996 OSA Technical Digest Series (Optical Society of America, 1996), paper ThG1.

\bibitem{steup96} D. Steup and J. Weinzierl, ``Resonant THz-meshes,''
presented at the Fourth International Workshop on THz Electronics, Erlangen-Tennenlohe, Germany, 5--6 Sept. 1996.

\end{thebibliography}


\end{document}
